\documentclass{article}
\usepackage[portuguese]{babel}
\usepackage[utf8]{inputenc}
\usepackage[T1]{fontenc}
\usepackage{oz} %Z notatio
\usepackage{listings}
\usepackage{verbatim}
\usepackage{hyperref}

\hypersetup{
    colorlinks=true,       % false: boxed links; true: colored links
    linkcolor=red,          % color of internal links (change box color with linkbordercolor)
    citecolor=green,        % color of links to bibliography
    filecolor=magenta,      % color of file links
    urlcolor=black           % color of external links
}

\title{SolveSurvo\\ Z3}
\author{José Pinheiro \texttt{pg23208@alunos.uminho.pt} 
		\\ Tiago Guimarães \texttt{pg22832@alunos.uminho.pt}}
		
\begin{document}
\maketitle
\newpage
\section{Survo}

\paragraph{} O Survo é um puzzle onde o objectivo é preencher uma tabela m * n com inteiros entre 1...M*N, de tal maneira que cada um desses números aparece apenas uma vez e a soma da linha e coluna é igual ao inteiro do fundo da tabela e do lado direito da tabela. Podem ser dados inteiros na tabela inicial para garantir a unicidade da solução e para fazer a tarefa mais fácil.

\section{Z3}

\paragraph{} Como SMT-solver estamos a usar o z3 da Microsoft. Escolhemos este smt-solver porque nós foi recomendado nas aulas de vfs. Como linguagem de interface com o z3 escolhemos o python (z3Py) devido a ser uma linguagem simples, eficaz e devido a mesma suportar listas por compreensão.

\section{solveSurvo}

\paragraph{} O solveSurvo é invocado com \$ solveSurvo.py <ficheiro\_input>, se conseguir resolver o puzzle devolve um ficheiro output com a solução do puzzle. Necessita que o z3 esteja instalado e conectado com o python.

\subsection{Ficheiros input}

\paragraph{} Para representar um tabuleiro de survo, escolhemos uma sintaxe muito simples.
A primeira linha é os valores que o as colunas tem de somar, e a segunda linha é os valores que as linhas tem de somar.
Para representar os pontos dados, temos a posição da matriz e o valor correspondente.
\paragraph{}
Exemplo :
\begin{verbatim}
27	16	10	25
30	18	30
1 2 6
2 1 8
3 3 3
\end{verbatim} 

\subsection{Ficheiros output}
\paragraph{} o SolveSurvo da como output um ficheiro com a representação da matriz obtida e os valores que as linhas e as colunas tem de somar.

\paragraph{}
Exemplo :
\begin{verbatim}
[7, 6, 5, 12]
[8, 3, 2, 5]
[12, 7, 3, 8]

Linhas   : [30, 18, 30]
Colunas  : [27, 16, 10, 25]
\end{verbatim}

\subsection{Codificação do Survo}
\paragraph{} Para codificar o puzzle, codificamos 4 propriedades do puzzle:
 \begin{itemize}
 \item \textbf{(A)} Cada célula tem valores entre 1..m*n;
 \item \textbf{(B)} Cada número é único no tabuleiro;
 \item \textbf{(C)} O somatório das linhas tem de ser igual ao valor na coluna final;
 \item \textbf{(D)} O somatório das colunas tem de ser igual ao valor na linha final.
 \end{itemize}

Sendo que X é a matriz que representa o tabuleiro, e cx\_size*cy\_size é tamanho da mesma então a sua codificação é:

\paragraph{(A)}
\begin{verbatim}
[ And(1 <= X[j][i], X[j][i] <= cx_size*cy_size) for i in range(cx_size) 
												for j in range(cy_size) ]
\end{verbatim}
\paragraph{(B)}
\begin{verbatim}
[ Distinct(X[j][i]) for i in range(cx_size) for j in range(cy_size) ]
\end{verbatim}
\paragraph{(C)}
\begin{verbatim}
[ cy[i] == sum(X[i]) for i in range(cy_size) ]
\end{verbatim}
\paragraph{(D)}
\begin{verbatim}
[ cx[j] == sum([ X[i][j] for i in range(cy_size) ])
             for j in range(cx_size) ]

\end{verbatim}

\end{document}